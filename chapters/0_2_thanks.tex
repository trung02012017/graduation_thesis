\documentclass[../main.tex]{subfiles}
\begin{document}

\begin{center}
    \textbf{Acknowledgement}\\
\end{center}
First and foremost, I would like to thank all teachers in Hanoi University of Science and Technology, especially who are in School of Information and Communication Technology for giving me rewarding knowledge and experience to complete this project.

 I also wish to express my sincere appreciation to Dr. Nguyen Binh Minh - my supervisor and Dr. Giang Nguyen from Institute of Informatics, Slovak Academy of Sciences (IISAS), who convincingly guided and encouraged me to be professional and do the right thing even when the road got tough. Without their persistent help, the goal of this thesis would not have been realized.

Finally, I must express my very profound gratitude to my parents and to my friends for providing me with unfailing support and continuous encouragement throughout my years of study and through the process of researching and writing this thesis. Thank you. 
\begin{center}
    \textbf{Abstract}\\
\end{center}
The Sea Lion Optimization (SLnO) algorithm have been shown to be competitive in searching for an optimal solution. This thesis proposes improvements to this algorithm that are based on the idea from Particle Swarm Optimization (PSO) algorithm, and opposition-based learning (OBL), forming an improved version of SLnO called ISLO. The ISLO is further compared with other well-known nature-inspired algorithms on 30 benchmark functions. The statistical results show that the ISLO significantly outperforms others on majority of functions in terms of accuracy and robustness. Furthermore, the work introduce a new model (ISLO-CFNN) that is a combination of ISLO and Cascade Feedforward Neural Network (CFNN). The proposed model tries to tackle the time-series forecasting problem deprived from auto-scaling demand in cloud computing. ISLO-CFNN's performance is tested by 4 time-series datasets, and is compared with several widely used deep learning models. The experiments show that ISLO-CFNN is very competitive results with compared models, proving the superior optimizing ability of ISLO in various problems ranging from theoretical functions to engineering applications.

\begin{table}[H]
\centering
\begin{tabular}{p{5cm} c}
\multicolumn{1}{c}{\textbf{}} & Student \\
\textbf{}                     & \small{Signature and Name}
\end{tabular}
\end{table}

\end{document}